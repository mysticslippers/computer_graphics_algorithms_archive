\documentclass[12pt]{report}

\usepackage{cmap}
\usepackage[T1,T2A]{fontenc}
\usepackage[utf8]{inputenc}
\usepackage[english, russian]{babel}
\usepackage{amssymb}
\usepackage{amsmath}
\usepackage{amsthm}
\usepackage{dsfont}
\usepackage{bm}
\usepackage{diagbox}
\usepackage{array}
\usepackage{placeins}
\usepackage[left=20mm,right=10mm,top=20mm,bottom=20mm,bindingoffset=2mm]{geometry}
\usepackage{indentfirst}
\usepackage[utf8]{inputenc}
\usepackage{float}
\usepackage[hidelinks]{hyperref}
\usepackage{graphicx}
\usepackage{xcolor}
\usepackage{listings}
\usepackage{minted}
\usepackage{enumitem}

\DeclareMathOperator{\N}{\mathbb{N}}
\DeclareMathOperator{\R}{\mathbb{R}}
\DeclareMathOperator{\Z}{\mathbb{Z}}
\DeclareMathOperator{\CC}{\mathbb{C}}
\DeclareMathOperator{\PP}{\mathrm{P}}
\DeclareMathOperator{\Expec}{\mathrm{E}}
\DeclareMathOperator{\Var}{\mathrm{Var}}
\DeclareMathOperator{\Cov}{\mathrm{Cov}}
\DeclareMathOperator{\asConv}{\xrightarrow{a.s.}}
\DeclareMathOperator{\LpConv}{\xrightarrow{Lp}}
\DeclareMathOperator{\pConv}{\xrightarrow{p}}
\DeclareMathOperator{\dConv}{\xrightarrow{d}}

\hypersetup{
	colorlinks=true,
	linkcolor=blue,
	citecolor=blue,
	urlcolor=blue
}

\lstset{language=Python, extendedchars=\true}

\lstdefinestyle{pythonstyle}{
	language=Python,
	backgroundcolor=\color{lightgray},
	commentstyle=\color{green},
	keywordstyle=\color{blue},
	stringstyle=\color{red},
	basicstyle=\ttfamily,
	frame=single,
	breaklines=true
}

\addto\captionsrussian{\renewcommand{\refname}{Список использованных источников}}

\begin{document}
	
	\begin{titlepage}
		\begin{center}
			\large{Федеральное государственное автономное образовательное учреждение высшего образования <<Национальный исследовательский университет ИТМО>>}
		\end{center}
		
		\vspace{15em}
		
		\begin{center}
			\huge{\textbf{Лабораторная работа №2}} \\
			\large{По дисциплине <<Алгоритмы компьютерной графики>>} \\
		\end{center}
		
		\vspace{2em}
		
		\begin{flushright}
			\textit{\large{Выполнил:}} \\
			\large{Студент группы P3306} \\
			\large{Михайлов Дмитрий} \\
			\large{Андреевич} \\
			
			\textit{\large{Преподаватель:}} \\
			\large{Потемин Игорь} \\
			\large{Станиславович}
		\end{flushright}
		
		\vspace{2cm}
		
		\begin{figure}[h]
			\centering
			\includegraphics[width=0.5\linewidth]{image.png}
		\end{figure}
		
		\begin{center}
			Санкт-Петербург \\
			2025 год
		\end{center}
	\end{titlepage}
	
	\tableofcontents
	\newpage
	
	\addcontentsline{toc}{section}{Общее описание приложения}
	\section*{Общее описание приложения}
	
	Приложение представляет собой настольное GUI-приложение для применения различных фильтров к изображениям.
	
	Используемые библиотеки:
	\begin{itemize}[noitemsep]
		\item \textbf{Tkinter}~--- для построения графического интерфейса;
		\item \textbf{Pillow (PIL)}~--- для загрузки, обработки и сохранения изображений.
	\end{itemize}
	
	Функционал приложения:
	\begin{itemize}[noitemsep]
		\item загрузка изображения с диска или автоматическая загрузка файла \texttt{test.png} из папки с программой;
		\item отображение исходного и обработанного изображения;
		\item применение фильтров:
		\begin{itemize}[noitemsep]
			\item Grey (оттенки серого),
			\item Blur (размытие),
			\item Contrast (контраст),
			\item Brightness (яркость),
			\item Invert Colors (инверсия цветов);
		\end{itemize}
		\item сохранение результата обработки в формате PNG.
	\end{itemize}
	\newpage
	
	\addcontentsline{toc}{section}{Архитектура приложения}
	\section*{Архитектура приложения}
	
	\begin{enumerate}[noitemsep]
		\item \textbf{Класс \texttt{ImageEditorApp}}~--- основной класс приложения, в котором:
		\begin{itemize}[noitemsep]
			\item инициализируются элементы интерфейса;
			\item хранятся ссылки на исходное и обработанное изображение;
			\item реализованы методы обработки и вспомогательные функции.
		\end{itemize}
		
		\item \textbf{Интерфейс управления (панель сверху)} содержит:
		\begin{itemize}[noitemsep]
			\item кнопки:
			\begin{itemize}[noitemsep]
				\item \texttt{Grey},
				\item \texttt{Blur},
				\item \texttt{Contrast},
				\item \texttt{Brightness},
				\item \texttt{Invert Colors},
				\item \texttt{Open Image},
				\item \texttt{Save PNG};
			\end{itemize}
			\item ползунки (слайдеры):
			\begin{itemize}[noitemsep]
				\item \texttt{Radius}~--- для фильтра размытия;
				\item \texttt{Factor}~--- для контраста;
				\item \texttt{Bright}~--- для яркости.
			\end{itemize}
		\end{itemize}
		
		\item \textbf{Область отображения}:
		\begin{itemize}[noitemsep]
			\item левая часть~--- метка \texttt{Исходное}, показывает уменьшенную копию загруженного изображения;
			\item правая часть~--- метка \texttt{Результат}, показывает результат последнего применённого фильтра либо заглушку до обработки.
		\end{itemize}
		
		\item \textbf{Методы обработки}:
		\begin{itemize}[noitemsep]
			\item \texttt{make\_gray}~--- перевод в оттенки серого;
			\item \texttt{apply\_blur}~--- размытие;
			\item \texttt{apply\_contrast}~--- изменение контраста;
			\item \texttt{apply\_brightness}~--- изменение яркости;
			\item \texttt{invert\_colors}~--- инверсия цветов;
			\item \texttt{save\_image}~--- сохранение результата.
		\end{itemize}
		
		\item \textbf{Вспомогательные методы}:
		\begin{itemize}[noitemsep]
			\item \texttt{\_load\_default\_image()}~--- попытка загрузить \texttt{test.jpg} при старте;
			\item \texttt{open\_image\_dialog()}~--- диалог выбора файла;
			\item \texttt{\_load\_image(path)}~--- безопасная загрузка изображения и подготовка превью;
			\item \texttt{\_preview\_size(image)} / \texttt{\_resize\_for\_preview(image)}~--- расчёт размеров и масштабирование под окно;
			\item \texttt{\_show\_left(img)}, \texttt{\_show\_right(img)}~--- обновление левой и правой картинок в интерфейсе;
			\item \texttt{\_ensure\_image\_loaded()}~--- проверка, что изображение загружено перед обработкой;
			\item \texttt{\_update\_result(img)}~--- сохранение результата и обновление правого превью.
		\end{itemize}
	\end{enumerate}
	\newpage
	
	\addcontentsline{toc}{section}{Визуализация данных}
	\section*{Подробное описание фильтров}
	
	\subsection*{1. Фильтр ``Grey'' (Оттенки серого)}
	
	\textbf{Принцип работы.}  
	Цель фильтра~--- преобразовать цветное изображение в оттенки серого.
	
	В простейшем случае яркость пикселя можно вычислить как среднее от каналов R, G и B:
	\[
	gray = \frac{R + G + B}{3}.
	\]
	
	В приложении используется готовая функция \texttt{ImageOps.grayscale}, которая переводит изображение в одноканальный режим \texttt{L}, а затем результат приводится обратно к \texttt{RGBA} с учётом альфа-канала исходного изображения.
	
	\textbf{Пример кода:}
	\begin{lstlisting}[language=Python]
		rgb = self.orig_image.convert("RGB")
		gray_l = ImageOps.grayscale(rgb)
		gray_rgba = gray_l.convert("RGBA")
		
		if self.orig_image.mode == "RGBA":
		alpha = self.orig_image.getchannel("A")
		gray_rgba.putalpha(alpha)
		
		self._update_result(gray_rgba)
	\end{lstlisting}
	
	\textbf{Особенности:}
	\begin{itemize}[noitemsep]
		\item при наличии прозрачности (формат RGBA) альфа-канал сохраняется;
		\item результат~--- изображение в оттенках серого с тем же уровнем прозрачности.
	\end{itemize}
	
	\textbf{Результат.}  
	Цветное изображение превращается в чёрно-белое, при этом прозрачные области остаются прозрачными.
	\newpage
	
	\subsection*{2. Фильтр ``Blur'' (Размытие)}
	
	\textbf{Принцип работы.}  
	Размытие реализовано с помощью \texttt{ImageFilter.GaussianBlur} из Pillow. Гауссово размытие смешивает каждый пиксель с его соседями, используя гауссово распределение: чем ближе соседние пиксели, тем больший вклад они вносят. Гауссово размытие можно представить как свёртку изображения с гауссовым ядром:
	
	\[
	I'(x, y) = \sum_{i=-r}^{r}\sum_{j=-r}^{r} I(x+i, y+j)\cdot G(i, j),
	\]
	
	где \(I(x,y)\) — исходное значение пикселя, \(I'(x,y)\) — размытое значение, \(r\) — радиус окна свёртки,
	а \(G(i,j)\) — гауссова функция:
	\[
	G(i, j) = \frac{1}{2\pi\sigma^2}\exp\left(-\frac{i^2 + j^2}{2\sigma^2}\right).
	\]
	Параметр \(\sigma\) определяет «силу» размытия и в практических реализациях обычно связан с выбранным радиусом.
	\vspace*{1em}
	
	\textbf{Параметры:}
	\begin{itemize}[noitemsep]
		\item радиус размытия задаётся ползунком \texttt{Radius} (0--20);
		\item чем больше радиус, тем сильнее размытие.
	\end{itemize}
	
	\textbf{Пример кода:}
	\begin{lstlisting}[language=Python]
		radius = int(self.radius_scale.get())
		blurred = self.orig_image.filter(ImageFilter.GaussianBlur(radius=radius))
		self._update_result(blurred)
	\end{lstlisting}
	
	\textbf{Результат.}  
	Изображение становится более «мягким», мелкие детали сглаживаются, текст и границы становятся менее резкими.
	\newpage
	
	\subsection*{3. Фильтр ``Contrast'' (Контрастность)}
	
	\textbf{Принцип работы.}  
	Контраст~--- это разница между тёмными и светлыми участками изображения. Фильтр регулирует эту разницу, делая изображение либо более «плоским», либо более «выразительным».
	
	\textbf{Общее математическое представление:}
	\[
	\text{новое\_значение} = (\text{старое\_значение} - \text{среднее}) \times \text{фактор} + \text{среднее},
	\]
	где:
	\begin{itemize}[noitemsep]
		\item \textit{среднее}~--- некая средняя яркость;
		\item \textit{фактор}~--- множитель контраста (в приложении от 0.0 до 3.0).
	\end{itemize}
	
	\textbf{Техническая реализация:}
	\begin{itemize}[noitemsep]
		\item используется класс \texttt{ImageEnhance.Contrast} из Pillow;
		\item обрабатываются только RGB-каналы;
		\item альфа-канал извлекается отдельно и затем объединяется с обработанными каналами.
	\end{itemize}
	
	\textbf{Пример кода:}
	\begin{lstlisting}[language=Python]
		factor = float(self.contrast_scale.get())
		
		rgba = self.orig_image.convert("RGBA")
		rgb = rgba.convert("RGB")
		alpha = rgba.getchannel("A")
		
		enhanced_rgb = ImageEnhance.Contrast(rgb).enhance(factor)
		r, g, b = enhanced_rgb.split()
		result = Image.merge("RGBA", (r, g, b, alpha))
		
		self._update_result(result)
	\end{lstlisting}
	
	\textbf{Влияние параметров:}
	\begin{itemize}[noitemsep]
		\item \texttt{factor = 0.0}~--- изображение становится полностью серым (всё стягивается к средней яркости);
		\item \texttt{factor = 1.0}~--- исходный контраст без изменений;
		\item \texttt{factor > 1.0}~--- контраст усиливается (тёмные участки темнеют, светлые светлеют);
		\item \texttt{factor < 1.0}~--- контраст уменьшается, изображение становится более «блеклым».
	\end{itemize}
	
	\textbf{Результат.}  
	Регулируется общая выразительность изображения: можно сделать его более ярким и драматичным или, наоборот, приглушённым.
	\newpage
	
	\subsection*{4. Фильтр ``Brightness'' (Яркость)}
	
	\textbf{Принцип работы.}  
	Фильтр изменяет общую яркость изображения, делая его темнее или светлее. Изменение яркости можно описать как умножение цветовых каналов на коэффициент \(k\):
	\[
	I'(x, y) = \mathrm{clip}\bigl(I(x, y)\cdot k\bigr),
	\]
	где \(k\) — фактор яркости (ползунок), а функция \(\mathrm{clip}(\cdot)\) ограничивает значение диапазоном допустимых
	интенсивностей (например, \([0, 255]\) для 8-битных каналов).
	При \(k>1\) изображение становится светлее, при \(0<k<1\) — темнее, при \(k=1\) не меняется.
	\vspace*{1em}
	
	\textbf{Техническая реализация:}
	\begin{itemize}[noitemsep]
		\item используется класс \texttt{ImageEnhance.Brightness};
		\item аналогично контрасту, RGB и альфа-канал обрабатываются раздельно.
	\end{itemize}
	
	\textbf{Пример кода:}
	\begin{lstlisting}[language=Python]
		factor = float(self.brightness_scale.get())
		
		rgba = self.orig_image.convert("RGBA")
		rgb = rgba.convert("RGB")
		alpha = rgba.getchannel("A")
		
		enhanced_rgb = ImageEnhance.Brightness(rgb).enhance(factor)
		r, g, b = enhanced_rgb.split()
		result = Image.merge("RGBA", (r, g, b, alpha))
		
		self._update_result(result)
	\end{lstlisting}
	
	\textbf{Интерпретация значений фактора:}
	\begin{itemize}[noitemsep]
		\item \texttt{factor = 0.0}~--- изображение становится полностью чёрным;
		\item \texttt{factor = 1.0}~--- исходная яркость без изменений;
		\item \texttt{factor > 1.0}~--- изображение осветляется;
	\end{itemize}
	
	\textbf{Результат.}  
	Изменяется общая освещённость изображения, что полезно для коррекции тёмных или пересвеченных фотографий.
	\newpage
	
	\subsection*{5. Фильтр ``Invert Colors'' (Инверсия цветов)}
	
	\textbf{Принцип работы.}  
	При инверсии каждый цвет заменяется на «противоположный» по формуле:
	\[
	\text{новый\_канал} = 255 - \text{старый\_канал}.
	\]
	
	Например:
	\begin{itemize}[noitemsep]
		\item чёрный (0, 0, 0) превращается в белый (255, 255, 255);
		\item белый превращается в чёрный;
		\item красный (255, 0, 0) превращается в голубой (0, 255, 255) и т.\,д.
	\end{itemize}
	
	\textbf{Реализация в коде.}  
	Вместо ручного прохода по пикселям используется функция \texttt{ImageOps.invert} для RGB-части изображения, при этом альфа-канал сохраняется:
	
	\begin{lstlisting}[language=Python]
		rgba = self.orig_image.convert("RGBA")
		rgb = rgba.convert("RGB")
		alpha = rgba.getchannel("A")
		
		inverted_rgb = ImageOps.invert(rgb)
		r, g, b = inverted_rgb.split()
		result = Image.merge("RGBA", (r, g, b, alpha))
		
		self._update_result(result)
	\end{lstlisting}
	
	\textbf{Особенности:}
	\begin{itemize}[noitemsep]
		\item прозрачность не меняется (альфа-канал не инвертируется);
		\item визуально эффект напоминает «фото-негатив».
	\end{itemize}
	
	\textbf{Результат.}  
	Получаем негатив исходного изображения с сохранением прозрачных областей.
	\newpage
	
	\addcontentsline{toc}{section}{Технические особенности}
	\section*{Технические особенности}
	
	\subsection*{Работа с альфа-каналом}
	
	Во всех фильтрах учитывается наличие прозрачности:
	\begin{itemize}[noitemsep]
		\item исходное изображение приводится к режиму \texttt{RGBA};
		\item альфа-канал (\texttt{getchannel("A")}) сохраняется отдельно;
		\item обработка (контраст, яркость, инверсия, серый) применяется только к цветовым каналам (RGB);
		\item в финальном результате каналы R, G, B объединяются с исходным альфа-каналом:
	\end{itemize}
	
	\begin{lstlisting}[language=Python]
		result = Image.merge("RGBA", (r, g, b, alpha))
	\end{lstlisting}
	
	Это позволяет корректно обрабатывать логотипы, иконки и другие изображения с прозрачным фоном.
	
	\subsection*{Масштабирование изображения}
	
	Для корректного отображения изображений в окне используется система превью:
	\begin{itemize}[noitemsep]
		\item максимальные размеры превью:
		\begin{itemize}[noitemsep]
			\item \texttt{PREVIEW\_MAX\_WIDTH = 500},
			\item \texttt{PREVIEW\_MAX\_HEIGHT = 400};
		\end{itemize}
		\item реальный размер превью вычисляется пропорционально:
	\end{itemize}
	
	\begin{lstlisting}[language=Python]
		ratio = min(
		PREVIEW_MAX_WIDTH / w,
		PREVIEW_MAX_HEIGHT / h,
		1.0,
		)
		new_size = (int(w * ratio), int(h * ratio))
	\end{lstlisting}
	
	Масштабирование выполняется с использованием высококачественного ресэмплинга:
	
	\begin{lstlisting}[language=Python]
		RESAMPLING = Image.Resampling.LANCZOS
		image.resize(new_size, RESAMPLING)
	\end{lstlisting}
	
	Обработка фильтрами всегда происходит над полной версией изображения, а не над уменьшенной копией.
	
	\subsection*{Загрузка и сохранение}
	
	\begin{itemize}[noitemsep]
		\item при запуске приложение пытается загрузить файл \texttt{test.jpg} из той же папки, что и скрипт; при отсутствии файла пользователю показывается сообщение об ошибке;
		\item кнопка \texttt{Open Image} вызывает диалог выбора файла с фильтрами по типам изображений;
		\item кнопка \texttt{Save PNG}:
		\begin{itemize}[noitemsep]
			\item сохраняет обработанное изображение;
			\item использует диалог \texttt{asksaveasfilename} с расширением \texttt{.png} по умолчанию;
			\item выводит сообщение об успехе или ошибке.
		\end{itemize}
	\end{itemize}
	\newpage
	
	\addcontentsline{toc}{section}{Пользовательский интерфейс}
	\section*{Пользовательский интерфейс}
	
	\begin{enumerate}[noitemsep]
		\item \textbf{Верхняя панель управления:}
		\begin{itemize}[noitemsep]
			\item 1-й ряд:
			\begin{itemize}[noitemsep]
				\item кнопка \texttt{Grey};
				\item кнопка \texttt{Blur};
				\item подпись \texttt{Radius:} и ползунок радиуса размытия;
			\end{itemize}
			\item 2-й ряд:
			\begin{itemize}[noitemsep]
				\item кнопка \texttt{Contrast};
				\item подпись \texttt{Factor:} и ползунок контраста;
				\item кнопка \texttt{Brightness};
				\item подпись \texttt{Bright:} и ползунок яркости;
			\end{itemize}
			\item 3-й ряд:
			\begin{itemize}[noitemsep]
				\item кнопка \texttt{Open Image};
				\item кнопка \texttt{Invert Colors};
				\item кнопка \texttt{Save PNG}.
			\end{itemize}
		\end{itemize}
		
		\item \textbf{Область изображений:}
		\begin{itemize}[noitemsep]
			\item слева~--- подпись \texttt{Исходное} и превью оригинала;
			\item справа~--- подпись \texttt{Результат} и превью обработанного изображения или заглушка до применения фильтра.
		\end{itemize}
	\end{enumerate}
	
	\begin{enumerate}[noitemsep]
		\item При запуске:
		\begin{itemize}[noitemsep]
			\item либо автоматически загружается \texttt{test.png};
			\item либо пользователь вручную выбирает изображение через \texttt{Open Image}.
		\end{itemize}
		\item Пользователь настраивает параметры фильтра (радиус, контраст, яркость) с помощью ползунков.
		\item Нажимает соответствующую кнопку фильтра (\texttt{Grey}, \texttt{Blur}, \texttt{Contrast}, \texttt{Brightness}, \texttt{Invert Colors}).
		\item Результат отображается справа.
		\item После получения нужного эффекта можно сохранить его с помощью \texttt{Save PNG}.
	\end{enumerate}
	
	\addcontentsline{toc}{section}{Заключение}
	\section*{Заключение}
	
	Разработанное приложение демонстрирует:
	\begin{itemize}[noitemsep]
		\item практическое использование библиотеки \textbf{Tkinter} для создания простого графического интерфейса;
		\item применение возможностей \textbf{Pillow} для обработки изображений:
		\begin{itemize}[noitemsep]
			\item использование встроенных фильтров (GaussianBlur, ImageEnhance);
			\item работу с цветовыми каналами и альфа-каналом;
			\item масштабирование изображений с сохранением пропорций.
		\end{itemize}
	\end{itemize}
	
	Приложение позволяет наглядно экспериментировать с параметрами фильтров (радиус размытия, контраст, яркость) и сразу видеть результат. Благодаря тому, что обработка всегда выполняется от исходного изображения, пользователь может многократно пробовать разные настройки, не накапливая артефакты предыдущих преобразований.
	
	
\end{document}

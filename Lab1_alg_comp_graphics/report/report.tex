\documentclass[12pt]{report}

\usepackage{cmap}
\usepackage[T1,T2A]{fontenc}
\usepackage[utf8]{inputenc}
\usepackage[english, russian]{babel}
\usepackage{amssymb}
\usepackage{amsmath}
\usepackage{amsthm}
\usepackage{dsfont}
\usepackage{bm}
\usepackage{diagbox}
\usepackage{array}
\usepackage{placeins}
\usepackage[left=20mm,right=10mm,top=20mm,bottom=20mm,bindingoffset=2mm]{geometry}
\usepackage{indentfirst}
\usepackage[utf8]{inputenc}
\usepackage{float}
\usepackage[hidelinks]{hyperref}
\usepackage{graphicx}
\usepackage{xcolor}
\usepackage{listings}
\usepackage{minted}

\DeclareMathOperator{\N}{\mathbb{N}}
\DeclareMathOperator{\R}{\mathbb{R}}
\DeclareMathOperator{\Z}{\mathbb{Z}}
\DeclareMathOperator{\CC}{\mathbb{C}}
\DeclareMathOperator{\PP}{\mathrm{P}}
\DeclareMathOperator{\Expec}{\mathrm{E}}
\DeclareMathOperator{\Var}{\mathrm{Var}}
\DeclareMathOperator{\Cov}{\mathrm{Cov}}
\DeclareMathOperator{\asConv}{\xrightarrow{a.s.}}
\DeclareMathOperator{\LpConv}{\xrightarrow{Lp}}
\DeclareMathOperator{\pConv}{\xrightarrow{p}}
\DeclareMathOperator{\dConv}{\xrightarrow{d}}

\hypersetup{
	colorlinks=true,
	linkcolor=blue,
	citecolor=blue,
	urlcolor=blue
}

\lstset{language=Python, extendedchars=\true}

\lstdefinestyle{pythonstyle}{
	language=Python,
	backgroundcolor=\color{lightgray},
	commentstyle=\color{green},
	keywordstyle=\color{blue},
	stringstyle=\color{red},
	basicstyle=\ttfamily,
	frame=single,
	breaklines=true
}

\addto\captionsrussian{\renewcommand{\refname}{Список использованных источников}}

\begin{document}
	
	\begin{titlepage}
		\begin{center}
			\large{Федеральное государственное автономное образовательное учреждение высшего образования <<Национальный исследовательский университет ИТМО>>}
		\end{center}
		
		\vspace{15em}
		
		\begin{center}
			\huge{\textbf{Лабораторная работа №1}} \\
			\large{По дисциплине <<Алгоритмы компьютерной графики>>} \\
		\end{center}
		
		\vspace{2em}
		
		\begin{flushright}
			\textit{\large{Выполнил:}} \\
			\large{Студент группы P3306} \\
			\large{Михайлов Дмитрий} \\
			\large{Андреевич} \\
			
			\textit{\large{Преподаватель:}} \\
			\large{Потемин Игорь} \\
			\large{Станиславович}
		\end{flushright}
		
		\vspace{2cm}
		
		\begin{figure}[h]
			\centering
			\includegraphics[width=0.5\linewidth]{image.png}
		\end{figure}
		
		\begin{center}
			Санкт-Петербург \\
			2025 год
		\end{center}
	\end{titlepage}
	
	\tableofcontents
	\newpage
	
	\addcontentsline{toc}{section}{Общее описание приложения}
	\section*{Общее описание приложения}
	
	Приложение представляет собой графический интерфейс для анализа цветовых каналов RGB на наборе изображений.
	
	Основные функции приложения:
	\begin{itemize}
		\item загрузка нескольких изображений из списка путей;
		\item вычисление средних значений каналов R, G, B;
		\item определение количества пикселей, где доминирует каждый канал;
		\item отображение результата в виде изображения и столбчатой диаграммы.
	\end{itemize}
	
	Используемые библиотеки:
	\begin{itemize}
		\item \textbf{Tkinter} --- графический интерфейс;
		\item \textbf{Pillow (PIL)} --- загрузка и масштабирование изображений;
		\item \textbf{NumPy} --- числовая обработка массивов пикселей;
		\item \textbf{Matplotlib} --- построение диаграмм.
	\end{itemize}
	\newpage
	
	\addcontentsline{toc}{section}{Архитектура приложения}
	\section*{Архитектура приложения}
	
	\begin{enumerate}
		\item \textbf{Класс \texttt{App}} \\
		Основной класс приложения, который:
		\begin{itemize}
			\item создаёт главное окно;
			\item подготавливает изображения и вычисленные данные;
			\item управляет переключением между изображениями;
			\item обновляет элементы интерфейса.
		\end{itemize}
		
		\item \textbf{Интерфейс} \\
		Включает:
		\begin{itemize}
			\item подпись с ФИО и группой;
			\item кнопку <<Следующая>> для перелистывания изображений;
			\item текстовую строку с выводом средних значений RGB;
			\item холст (\texttt{Canvas}) для отображения уменьшенной копии изображения;
			\item область со встроенным графиком Matplotlib.
		\end{itemize}
		
		\item \textbf{Модуль анализа изображений} \\
		Для каждого изображения:
		\begin{itemize}
			\item выполняется загрузка и перевод в формат RGB;
			\item изображение преобразуется в массив NumPy;
			\item вычисляются средние значения каналов:
			\begin{lstlisting}[language=Python]
				r_mean, g_mean, b_mean = arr.mean(axis=(0, 1))
			\end{lstlisting}
			\item выделяются отдельные каналы и формируются маски доминирования:
			\begin{lstlisting}[language=Python]
				r_dom = (r > g) & (r > b)
				g_dom = (g > r) & (g > b)
				b_dom = (b > r) & (b > g)
			\end{lstlisting}
			\item подсчитывается количество пикселей по каждому доминирующему каналу:
			\begin{lstlisting}[language=Python]
				r_cnt = int(r_dom.sum())
				g_cnt = int(g_dom.sum())
				b_cnt = int(b_dom.sum())
			\end{lstlisting}
		\end{itemize}
	\end{enumerate}
	
	Результаты (средние значения и количества пикселей) сохраняются в списке, чтобы при переключении изображений не пересчитывать их заново.
	\newpage
	
	\addcontentsline{toc}{section}{Визуализация данных}
	\section*{Визуализация данных}
	
	Для отображения в интерфейсе каждое изображение масштабируется до фиксированного размера:
	\begin{lstlisting}[language=Python]
		TARGET_SIZE = (400, 250)
		view = src.resize(TARGET_SIZE, Resampling.LANCZOS)
	\end{lstlisting}
	
	Анализ выполняется по исходному изображению, а в окне показывается уменьшенная версия. Это позволяет сохранить точность расчётов и при этом не перегружать интерфейс.
	
	\addcontentsline{toc}{section}{Пользовательский интерфейс}
	\section*{Пользовательский интерфейс}
	
	\begin{itemize}
		\item \textbf{Верхняя часть:}
		\begin{itemize}
			\item надпись с ФИО и группой;
			\item кнопка <<Следующая>>;
			\item строка с текстом вида \\[2pt]
			\texttt{Среднее RGB: (R, G, B)}.
		\end{itemize}
		
		\item \textbf{Центральная часть:}
		\begin{itemize}
			\item холст \texttt{Canvas} с текущим изображением.
		\end{itemize}
		
		\item \textbf{Нижняя часть:}
		\begin{itemize}
			\item область с графиком Matplotlib (столбчатая диаграмма R/G/B).
		\end{itemize}
	\end{itemize}
	\newpage
	
	\addcontentsline{toc}{section}{Принцип работы}
	\section*{Принцип работы}
	
	\begin{enumerate}
		\item При запуске отображается первое изображение из списка.
		\item Пользователь видит:
		\begin{itemize}
			\item само изображение;
			\item средние значения каналов RGB;
			\item диаграмму доминирующих пикселей.
		\end{itemize}
		\item При нажатии на кнопку <<Следующая>>:
		\begin{itemize}
			\item отображается следующее изображение;
			\item обновляются средние значения и диаграмма;
			\item после последнего изображения просмотр продолжается с первого (циклический режим).
		\end{itemize}
	\end{enumerate}
	
	\addcontentsline{toc}{section}{Заключение}
	\section*{Заключение}
	
	Приложение демонстрирует работу с графическим интерфейсом на базе Tkinter, загрузку и обработку изображений с помощью библиотеки Pillow, использование NumPy для статистического анализа по пикселям, а также интеграцию Matplotlib в окно Tkinter для наглядной визуализации. В результате реализован простой и наглядный инструмент для RGB-анализа изображений, позволяющий оценивать средние значения цветовых каналов и распределение доминирующих цветов в удобной графической форме.
	
\end{document}
